\documentclass[../main.tex]{subfiles}

\begin{document}

\section{Richiami sulle Reti a Pacchetti}
\label{sec:reti}

\subsection{Concetti di Base}
Una rete di telecomunicazione è un insieme di nodi connessi da canali di comunicazione, con lo scopo di trasportare informazione da una sorgente a una destinazione.
In una \textbf{rete a pacchetto}, l'informazione viene suddivisa in pacchetti, che viaggiano attraverso la rete, potenzialmente seguendo strade differenti. \textbf{Internet} è un esempio di rete a pacchetto, definita come un'unione di sottoreti tra loro interconnesse.

Esistono due livelli di comunicazione:
\begin{itemize}
    \item \textbf{Comunicazione Logica:} La comunicazione come vista dagli applicativi (es. dalla webcam allo smartphone).
    \item \textbf{Comunicazione Fisica:} Il percorso reale che i dati compiono attraverso la rete, passando di nodo in nodo (switch, router, ecc.) fino alla destinazione.
\end{itemize}

\subsection{Pacchettizzazione e Indirizzamento}
Il flusso di dati prodotto da un'applicazione (es. un video compresso) viene "pacchettizzato", ovvero spezzato in pezzetti più piccoli. Questo processo introduce due problematiche fondamentali:
\begin{enumerate}
    \item \textbf{Riordino:} I pacchetti possono viaggiare su percorsi diversi e arrivare fuori ordine. È necessario un \emph{numero di sequenza} per riordinarli a destinazione.
    \item \textbf{Indirizzamento:} I pacchetti devono essere indirizzati correttamente.
\end{enumerate}

Per gestire l'indirizzamento, vengono aggiunte informazioni (header) a due livelli principali:
\begin{itemize}
    \item \textbf{Header di Trasporto:} Specifica l'applicazione sorgente (N) e l'applicazione destinazione (M) sul nodo. Questi identificativi sono detti \textbf{porte}.
    \item \textbf{Header di Rete:} Specifica il nodo sorgente ($addr_{src}$) e il nodo destinazione ($addr_{dst}$) sulla rete globale. Questi sono gli \textbf{indirizzi IP}.
\end{itemize}

\subsection{Stack Protocollare TCP/IP}
Questo processo di aggiunta di header (incapsulamento) segue un approccio a livelli (o strati), noto come "divide et impera". L'architettura standard di Internet è la \textbf{TCP/IP}, formalizzata in 5 livelli.

%%% TODO: Inserire immagine: Architettura TCP/IP 5 livelli (Pag 24)

\begin{description}
    \item[Livello 5: Applicativo]
    Fornisce i servizi all'utente (es. Web, E-mail, Messaggistica). I dati prodotti qui sono il "payload" per il livello sottostante.

    \item[Livello 4: Trasporto]
    Fornisce un canale di trasporto \emph{end-to-end} (tra le due applicazioni). Introduce l'header di trasporto (con le porte). Offre due protocolli principali:
    \begin{itemize}
        \item \textbf{TCP (Transmission Control Protocol):} Orientato alla connessione e \emph{affidabile}. Gestisce il riordino, il recupero degli errori (ritrasmissione) e il controllo di flusso/congestione.
        \item \textbf{UDP (User Datagram Protocol):} \emph{Connectionless} e \emph{non affidabile} (best-effort). Non imposta una connessione e non ritrasmette i pacchetti persi.
    \end{itemize}

    \item[Livello 3: Rete]
    Gestisce il trasferimento dei pacchetti tra nodi \emph{su tutta la rete} (internetworking). Il protocollo è \textbf{IP (Internet Protocol)}.
    \begin{itemize}
        \item Aggiunge l'header di rete, che contiene l'indirizzo IP sorgente e destinazione (es. 192.168.0.0/16 per reti private).
        \item Il servizio IP non è affidabile e non garantisce l'ordine: i pacchetti possono essere persi, duplicati o alterati (best-effort).
        \item L'instradamento (\textbf{routing}) è l'operazione chiave di questo livello. Ogni router prende una decisione locale ("hop-by-hop") basandosi sulla propria tabella di routing per inoltrare il pacchetto al nodo successivo.
    \end{itemize}

    \item[Livello 2: Data-link]
    Fornisce un canale di comunicazione affidabile tra macchine \emph{adiacenti} (sullo stesso cavo o AP).
    \begin{itemize}
        \item \textbf{Framing:} Incapsula il pacchetto IP in una "trama" (frame), aggiungendo un header (H) e un trailer (T).
        \item \textbf{Indirizzamento Fisico:} Usa un indirizzo fisico univoco detto \textbf{MAC} (es. AA:BB:CC:DD:EE:FF) per identificare i nodi sulla LAN (es. Ethernet, Wi-Fi).
        \item \textbf{Controllo d'Errore:} Il trailer contiene un \emph{checksum} per rilevare se la trama è stata corrotta durante la trasmissione.
        \item \textbf{Accesso al Mezzo:} Gestisce la condivisione del canale (es. CSMA/CA per Wi-Fi).
    \end{itemize}

    \item[Livello 1: Fisico]
    Gestisce la trasmissione del singolo flusso di bit sul canale fisico (impulsi elettrici su rame, luce su fibra, onde radio).
\end{description}

\subsection{Modello di Comunicazione: Host vs Router}
Il funzionamento dei livelli è diverso tra un nodo terminale (Host) e un nodo intermedio (Router):
\begin{itemize}
    \item Un \textbf{Host} (sorgente o destinazione) implementa tutti e 5 i livelli dello stack TCP/IP.
    \item Un \textbf{Router} implementa solo i livelli 1, 2 e 3 (Fisico, Data-link, IP). Il suo compito è ricevere una trama L2, estrarre il pacchetto L3 (IP), consultare la tabella di routing, e re-incapsularlo in una nuova trama L2 per inviarlo all'hop successivo.
\end{itemize}

%%% TODO: Inserire immagine: Stack protocollare Host vs Router (Pag 51)

\end{document}
