\documentclass[a4paper,12pt]{article}

\usepackage[utf8]{inputenc}
\usepackage[T1]{fontenc}
\usepackage[italian]{babel}

\usepackage{amsmath, amssymb, amsfonts}

\usepackage[a4paper, margin=2.5cm, top=2.2cm, bottom=2.5cm]{geometry}
\setlength{\parskip}{0.6em}    
\setlength{\parindent}{0pt}    
\usepackage{changepage}
\usepackage{subfiles}
\usepackage{titling}    
\usepackage{multicol}

\usepackage{graphicx}
\graphicspath{{figures/}}
\usepackage{subcaption}        
\usepackage{float}             

\usepackage[dvipsnames]{xcolor}
\usepackage[
    colorlinks=true,
    linkcolor=NavyBlue,
    urlcolor=Magenta,
    citecolor=OliveGreen
]{hyperref}


\usepackage{fancyhdr}
\pagestyle{fancy}
\fancyhf{} 
\fancyhead[L]{\textit{Appunti di Cybersecurity}}
\fancyhead[R]{\textit{Corso ITID}}
\fancyfoot[C]{\thepage}

\usepackage{csquotes}
\usepackage{enumitem}
\setlist{itemsep=0.4em, topsep=0.3em} 

\usepackage{lmodern}
\linespread{1.15}

\title{\vspace{-2em}\textbf{\Huge Appunti di Cybersecurity}\vspace{-0.5em}}
\author{\Large Andrea Bellu}
\date{\today}

\pretitle{\begin{center}\LARGE\bfseries}
\posttitle{\par\end{center}\vspace{0.5em}}
\preauthor{\begin{center}\large}
\postauthor{\par\end{center}}
\predate{\begin{center}\large}
\postdate{\par\end{center}}

\begin{document}

\noindent
\maketitle

{\scriptsize Questi appunti sono stati compilati con l'aiuto di Gemini. Possono contenere errori.}

\tableofcontents
\newpage

\section{Introduzione alla Cybersecurity}

\subsection{Perché Cybersecurity? Il Contesto di Industria 4.0}
Il bisogno di cybersecurity è cresciuto esponenzialmente con l'arrivo dell'\textbf{Industria 4.0} (ora "Impresa 4.0"). Questo paradigma si basa su tecnologie abilitanti che interconnettono l'intera filiera produttiva.
\begin{itemize}
    \item Questa interconnessione (es. feedback dalle vendite alla produzione) crea \textbf{molte nuove possibilità per fare danni}.
    \item L'elemento fondamentale è la necessità di \textbf{meccanismi per garantire la sicurezza nelle comunicazioni}.
\end{itemize}

\subsection{Esempi di CyberThreats}
Le minacce informatiche (cyberthreats) dimostrano la necessità di protezione in vari ambiti:
\begin{itemize}
    \item \textbf{Robot Industriali}:
          \begin{itemize}
              \item Attacco tramite smartphone infetto che, una volta nella rete aziendale, si finge il repository per gli aggiornamenti.
              \item Il robot scarica così codice malevolo.
              \item \textbf{Problema di fondo:} Rete aziendale per i client non separata dalla rete di produzione.
              \item \textbf{Danno subdolo:} Modificare il movimento di pochi mm, creando migliaia di pezzi difettosi.
          \end{itemize}

    \item \textbf{Stuxnet}:
          \begin{itemize}
              \item Virus sviluppato da governi (US contro Iran).
              \item È entrato nella rete isolata della centrale nucleare di Natanz tramite una \textbf{chiavetta USB}.
              \item \textbf{Obiettivo:} Attaccare i PLC di controllo delle centrifughe (Siemens).
          \end{itemize}

    \item \textbf{Domino's Pizza}:
          \begin{itemize}
              \item L'app per smartphone gestiva tutto il processing, \textbf{senza un double-check lato server}.
              \item Era possibile inviare un ordine alla produzione senza aver pagato.
              \item \textbf{Lezione:} L'app deve essere un'interfaccia, il processing va fatto sul server.
          \end{itemize}

    \item \textbf{Chipset Bluetooth Broadcom}:
          \begin{itemize}
              \item È stato scoperto un comando di debug via wireless.
              \item Esempio di fallimento della "Security by Obscurity".
          \end{itemize}

    \item \textbf{WhatsApp (CVE-2019-3568)}:
          \begin{itemize}
              \item Vulnerabilità di tipo \textbf{Buffer Overflow} durante l'instaurazione di una sessione VoIP.
              \item Inviando un messaggio "opportuno" (più lungo del buffer), l'attaccante poteva sovrascrivere altre parti di memoria.
              \item \textbf{Danno:} Installare software simile a Pegasus per spiare l'utente (camera, microfono, ecc.).
          \end{itemize}

    \item \textbf{Virgin Media O2}:
          \begin{itemize}
              \item Configurazione errata del software IMS (IP Multimedia Subsystem).
              \item Informazioni sensibili (Cell ID, IMSI/IMEI) venivano incluse negli \textbf{header del protocollo SIP}.
              \item \textbf{Danno:} Permetteva a un attaccante di mappare la posizione geografica degli utenti.
          \end{itemize}
\end{itemize}

\subsection{Il fattore umano e la convergenza delle reti}
Molti problemi di sicurezza sono legati alla "pigrizia" (*laziness*):
\begin{itemize}
    \item Protocolli senza autenticazione.
    \item Paradigma "security by obscurity" (es. GSM).
    \item Password fisse (magari appuntate su una lavagna).
    \item Mancanza di budget/tempo per aggiornare hw/sw (WannaCry docet).
\end{itemize}

\section{Reti a Pacchetti e Convergenza}

\subsection{Concetti Base delle Reti a Pacchetti}
\begin{itemize}
    \item Una rete a pacchetto è un insieme di nodi connessi da canali, dove l'informazione è divisa in pacchetti.
    \item Internet è un'unione di sottoreti tra loro interconnesse.
    \item \textbf{Concetto Chiave:} Nello schema originale di Internet, i nodi intermedi non offrono \textbf{nessun servizio di sicurezza}.
    \item Tutta la sicurezza deve essere gestita \textbf{dai nodi terminali} (principio "end-to-end").
\end{itemize}

\subsection{Evoluzione e Convergenza delle Reti}
\begin{itemize}
    \item \textbf{Reti LAN:} Si è passati da reti cablate (switch) a reti wireless (Access Point). Questo introduce il rischio di "ascolto" (*eavesdropping*).
    \item \textbf{Reti Industriali:} Si è passati da controlli punto-punto (cablaggio complesso) a Bus di campo (Fieldbus) e infine a \textbf{sistemi a pacchetto} (es. Profinet, Wi-Fi).
    \item \textbf{Convergenza:} Oggi, i problemi di sicurezza sono gli stessi ovunque (casa, azienda, produzione) a causa dell'uso delle stesse tecnologie di rete.
\end{itemize}

\section{Meccanismi e Pilastri della Sicurezza}

\subsection{I Servizi di Sicurezza (Pilastri)}
Per proteggere le comunicazioni servono "servizi" specifici:
\begin{enumerate}
    \item \textbf{Autenticazione:} Identificare \emph{chi} partecipa alla comunicazione (utente, nodo, applicazione) chiedendo una "prova".
    \item \textbf{Controllo di Accesso:} Verificare i \emph{diritti} di un partecipante (già autenticato) ad accedere a una risorsa. È \emph{successivo} all'autenticazione.
    \item \textbf{Confidenzialità:} Garantire che solo chi è autorizzato possa \emph{leggere} le informazioni (sia in transito che memorizzate).
    \item \textbf{Integrità (Integrity Protection):} Garantire che chi non è autorizzato non possa \emph{modificare} l'informazione senza essere scoperto.
    \item \textbf{Non Ripudio:} Garantire che un'entità non possa \emph{negare} in seguito di aver generato un'informazione o partecipato a un processo.
\end{enumerate}

\subsection{Crittografia}
\begin{itemize}
    \item La crittografia è il "blocco fondamentale" per ottenere meccanismi di sicurezza "forti".
    \item Include sistemi a chiave simmetrica (private key), asimmetrica (public key), Hash e MAC.
    \item \textbf{Principio di Kerckhoffs:} La sicurezza di un sistema deve dipendere dalla \textbf{segretezza della chiave}, e non dalla segretezza del sistema/algoritmo (come invece si pensava per la macchina Enigma).
\end{itemize}

\subsection{Tipi di Attacchi}
\begin{itemize}
    \item \textbf{Attacchi Passivi:} L'attaccante può solo catturare e analizzare i dati (es. intercettazione).
    \item \textbf{Attacchi Attivi:} L'attaccante può ricevere, \emph{modificare} e re-immettere i dati in rete.
\end{itemize}

\section{Conclusioni: Relatività e Standard}
\begin{itemize}
    \item \textbf{La Sicurezza Assoluta Non Esiste:} È sempre \emph{relativa} all'ambito applicativo e al \emph{valore} di ciò che si protegge.
    \item \textbf{Compromesso:} Ogni meccanismo è un compromesso tra costo, complessità e livello di protezione.
    \item \textbf{Sicurezza di Sistema vs. di Rete:} La prima riguarda il singolo nodo (HW, SW, OS), la seconda la comunicazione \emph{tra} i nodi (protocolli).
    \item \textbf{Normative (EU):} Dal 2022 in Europa è presente la \textbf{NIS2} (Network and Information Security 2). Ha l'obiettivo di aumentare il livello di sicurezza in settori critici (energia, sanità, PA, ecc.) imponendo obblighi di gestione del rischio.
    \item \textbf{Framework:} Il \textbf{NIST Cybersecurity Framework (CSF)} è un esempio di approccio sistematico per gestire i rischi (e può essere usato per adeguarsi alla NIS2).
\end{itemize}

\subfile{sections/reti}

\subfile{sections/cypher-intro}

\subfile{sections/symm}

\subfile{sections/asymm}

\end{document}
